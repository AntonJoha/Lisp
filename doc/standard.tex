
\section{Standard library}

This chapter is meant to document all of the different functions that reside in the standard library. 
There is no real plan for what is to be added and when. 
But when new features arrive they should be added here. 
There's an issue with the intent of storing different function needs and ideas found \href{https://github.com/AntonJoha/Lisp/issues/11}{here}.
If anyone happens to find this then feel free to add to it. 


\subsection{User interface}

This is a collection of predefined functions which are to be used by the users. 
This will range from functions for user input and output. 
It might also in the future contain functions for basic TUI. 


\subsubsection{print}

This is a function which takes values and prints them to the terminal.

Usage:
\begin{lstlisting}
(print "Hello world")
\end{lstlisting}
Prints ``Hello world'' to the terminal.
\begin{lstlisting}
(print '(1 2 3))
\end{lstlisting}
Prints ``(1 2 3)'' to the terminal.
\begin{lstlisting}
(print '(1 2 3) '(4 5 6))
\end{lstlisting}
Prints ( 1 2 3 ) ( 4 5 6 ) to the terminal.

\subsubsection{input}

This is a function which takes an input from the user in the form of a line. 
It then returns a string of the line. 
It will always be a string type.

Usage:
\begin{lstlisting}
(input)
\end{lstlisting}

\subsection{List operations}

This is a collection over all of the predefined operations which can be done on a list.

\subsubsection{list}

This is the function which can be used to create a list. 
It takes an unspecified amount of variables and returns a list of them in the given order. 

Usage:
\begin{lstlisting}
(list 1 2 3 4 5)
\end{lstlisting}
Creates the list (1 2 3 4 5).
\begin{lstlisting}
( def calc ( left right) 
	( list
		( + left right)
		( - left right)
		( * left right)
		( / left right)
	)
)
\end{lstlisting}
Definition for a function which creates a list of the sum, difference, product and quotient of two numbers.
\begin{lstlisting}
(list ( calc 1 2) ( calc 3 4))
\end{lstlisting}
A list of lists.

\subsubsection{rest}

This function take a list and returns another list with all but the first element. 
It's intended to used for recursion and other situations. 

Usage:
\begin{lstlisting}
(rest ( list 1 2 3 4 5))
\end{lstlisting}
Returns (2 3 4 5).

\subsubsection{first}

This is a function which takes a list and returns the first element of it.

Usage:
\begin{lstlisting}
(first '( 1 2 3 4 5))
\end{lstlisting}
Returns 1.

\subsubsection{get}

This function takes a list and an index and returns the element at that index.

Usage:
\begin{lstlisting}
(get 2 '(1 2 3 4 5) )
\end{lstlisting}
Returns 3.


\subsection{Type operations}

This is a collection of functions which regard the type of a variable. 
It can be used to change type, check type or more. 

\subsubsection{integer} \label{type:integer}

Convert one or more variables to an integer.

Usage:
\begin{lstlisting}
(integer 1.0)
\end{lstlisting}
Returns 1.
\begin{lstlisting}
(integer 1.0 2.0 3.0)
\end{lstlisting}
Returns (1 2 3).
\begin{lstlisting}
(integer "1")
\end{lstlisting}
Returns 1.
\begin{lstlisting}
(integer '( "1" "2" ) 3.0)
\end{lstlisting}
Returns ( ( 1 2 )  3).

\subsubsection{float}

Like integer but returns float instead see \ref{type:integer}. 


\subsubsection{string}

Like integer but returns string instead see \ref{type:integer}.

\subsubsection{id}

Like integer but returns input instead see \ref{type:integer}.
