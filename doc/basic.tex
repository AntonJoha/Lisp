\section{Language}


A language has to be defined before anything else, no language nothing to implement. 
The specification is going to be under construction for a good while (if it ever is done).
Reason for this being that I might make a suboptimal solution that needs to be swapped out.I plan on structuring the language by taking the parts of different languages that I like the most, without making it messy. 



\subsection{Lists}


The list handling will be heavily inspired by how it's described in the GNU Emacs manual\cite{fsf:22}. 
Flow handling, like the calling of functions and such is done withing two parentheses.

\begin{lstlisting}[caption=Example of how the function list looks]
(+ 1 2)
\end{lstlisting}

The first entry in this list is the function to be called, the following values are arguments to this function. 
The arguments can themselves include lists, leading to interesting results. 
\begin{lstlisting}[caption=Nesting of functions!]
(+ (+ 1 2) (+ 3 4))
\end{lstlisting}

There are also lists which are not evaluated. 
These inspired by \cite{fsf:22} are declared with the character `` ' ''.

\begin{lstlisting}[caption=Example of how a list with only plain values look]
'(a b c)
\end{lstlisting}

These two in combination provide a way through which functions can be called, and also a way through which plain lists are declared. 
Both of them can then be combined as seen in listing \ref{listing:listexample}

\begin{lstlisting}[caption=Function list with a pure list as one of the arguments,
	label=listing:listexample]
(fun '(a b c) '(d e f) )
\end{lstlisting}


\subsection{Keywords}

The programming language contains some keywords which have different functions.
A keyword is a reserved word in the language, meaning that nothing can have the same name as it. 
All keywords currently defined is listed below each in it's own subsection.
Do note that this is not a definit list, new keywords might be added or renamed.

\subsubsection{def}

\texttt{def} is the keyword used to define functions.
Listings \ref{listing:defexample} shows how this is to be used.
The keyword def is used followed by the name of the function to be defined. 
It then expects a list which is the amount of arguments given.
The defined function would then be called as shown in listing \ref{listing:defexamplecall}

\begin{lstlisting}[caption=Example of how the keyword def is to be used,
	label=listing:defexample]
(def adding (arg1 arg2)
	(+ arg1 arg2)
)
\end{lstlisting}

\begin{lstlisting}[caption=Calling a function named adding with two integers as argument,
	label=listing:defexamplecall]
(adding 4 9)
\end{lstlisting}


\subsection{Variables}

Starting off only four different kinds of basic types will be used: integers, floats, char and string. 
More might be added at a later stage.

Boolean will not be it's own type, it will instead be handled similar to how it's done in C.
Any non-zero value will therefore be regarded as true, while zero is treated as false. 

\subsubsection{Operators}

A language typically include different kinds of operators which the data types can be manipulated with, this one is no difference.
This however requires a defined system of what should happen with each combination of operator. 

The operators currently defined are:
\begin{itemize}
	\item \texttt{+}
	\item \texttt{-}
	\item \texttt{==}
	\item \texttt{<}
	\item \texttt{*}
	\item \texttt{/}
	\item \texttt{!}
	\item \texttt{\&\&}
	\item \texttt{||}
	\item More to be added here
\end{itemize}

More operators exist, but this is at the moment kept to a minimum. 
This is because having more operators would mean more rules to define. 
It is therefore beneficial to include redundant operators. 
For example the operator \texttt{>} i.e is not needed as \texttt{<} exists.
The function can instead be defined in code in terms of the other as shown in listing \ref{listing:biggerthandefined}

\begin{lstlisting}[caption=Bigger than function defined in terms of less than,
label=listing:biggerthandefined]
(def > (arg1 arg2)
	(< arg2 arg1)
)
\end{lstlisting}

The same can be done for other functions, more examples are shown below.

\begin{lstlisting}[caption=Less than or equal to defined with the OR operator]
(def <= (arg1 arg2)
	(|| 
		(< arg1 arg2)
		(== arg1 arg2)
	)
)
\end{lstlisting}

\begin{lstlisting}[caption=More than or equal to defined with less than or equal to]
(def >= (arg1 arg2)
	(<= arg2 arg1) 
)
\end{lstlisting}

\begin{lstlisting}[caption=Definition for is not equal with the invert operator]
(def != (arg1 arg2)
	(! (== arg1 arg2))
)
\end{lstlisting}


\section{Grammar}

This is a grammar presented in a BNF. 


\begin{lstlisting}
expression-list ::= <expression> <expression-list> | <expression>
expression ::= <list> 
	| <evals>
list ::= '( <entry-list> )
evals ::= ( <func> <entry-list> ) 
func ::= <operator> | <pre-defined>
id ::= [a-z,A-Z][a-z,A-Z,0-9,\_]+
	| <operator> 
<entry> ::= <number> | <id> | <expression-list>
<number> ::= <integer> | <float> 
<integer> ::= [0-9]+
<float> ::= [0-9]+ . [0-9]
<entry-list> ::= <entry> <entry-list> | empty
operator ::= *
	| +
	| -
	| !
	| < 
	| ||
	| ==
	| &&
	| /
\end{lstlisting}

